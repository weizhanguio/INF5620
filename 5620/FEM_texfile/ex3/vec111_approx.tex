\documentclass{article}
\usepackage{amsmath}
\usepackage{listings}
\usepackage{amssymb}
\usepackage{color}
\usepackage{graphicx}
\parindent=19pt
\begin{document}
\title{Finite Element Method}
\author{Wei Zhang}
\large
\maketitle

\section{Exercise 3: Approximate a three-dimensional vector in a plane}
We extend the unit vectors $\boldsymbol{\varphi}_{0}=(1,0)$ and $\boldsymbol{\varphi}_{1}=(0,1)$ to (1, 0, 0) and (0, 1, 0) respectively. The vector $\textbf{u}=\displaystyle \sum^{1}_{i=0}c_{i}\boldsymbol{\varphi}_{i}$, the error $\textbf{e}=\textbf{f}-\textbf{u}$. We should choose coefficients $c_{0}$, $c_{1}$ to get the best approximation. According to the Galerkin method, we obtain
\begin{equation}
(\textbf{e},\boldsymbol{\varphi}_{i})=0
\end{equation}
\begin{equation*}
\begin{cases}
(\textbf{f}-c_{0}\boldsymbol{\varphi}_{0}-c_{1}\boldsymbol{\varphi}_{1},\boldsymbol{\varphi}_{0})=0\\
(\textbf{f}-c_{0}\boldsymbol{\varphi}_{0}-c_{1}\boldsymbol{\varphi}_{1},\boldsymbol{\varphi}_{1})=0
\end{cases}
\end{equation*}
\begin{equation*}
\begin{cases}
(1-c_{0},1-c_{1},1)\cdot(1, 0, 0)=0\\
(1-c_{0},1-c_{1},1)\cdot(0, 1, 0)=0
\end{cases}
\end{equation*}
$c_{0}=1$, $c_{1}=1$. $\textbf{u}=(1, 1)$. \par
 The same principle applies to vectors (2,1) and (1,2). We get
\begin{equation*}
\begin{cases}
(1-2c_{0}-c_{1}, 1-c_{0}-2c_{1}, 1)\cdot(2, 1, 0)=0\\
(1-2c_{0}-c_{1}, 1-c_{0}-2c_{1}, 1)\cdot(1, 2, 0)=0
\end{cases}
\end{equation*}
$c_{0}=\displaystyle\frac{1}{3}$, $c_{1}=\displaystyle\frac{1}{3}$. $\textbf{u}=(1, 1)$.
\end{document}

